\documentclass [a4paper]{article}

%\usepackage[latin1]{inputenc}
\usepackage[utf8]{inputenc}
\usepackage{color}
\usepackage[french]{babel}
%\usepackage[francais]{babel}
\usepackage{rotating}
\usepackage {color}
\usepackage {latexsym}
\usepackage {graphicx, epsfig}
\usepackage{amsmath}
\usepackage{amsfonts}
\usepackage{supertabular}
\usepackage{lscape}
\usepackage{array}
\usepackage{longtable}
\usepackage{color}
\usepackage{colortbl}

%\addtolength{\textwidth}{1.5cm}
%\addtolength{\oddsidemargin}{-2.7cm}
%\addtolength{\topmargin}{-3.5cm}
%\addtolength{\botmargin}{1cm}
%\textwidth = 500pt
%\textheight=730pt \empty

\usepackage{geometry}
\geometry{a4paper, top=1.3cm, bottom=1.8cm, left=1.5cm, right=1.5cm}

%%%%%%%%%%%%%%%%%%%%%%%%%%%%%%%%%%%%%%%%%%%%%%%%%%%%%%%%%%%%%%%%%%%%%%%%%%%%%%%
%%%%%%%%%%%%%%%%%%%%%%%%%%%%%%%%%%%%%%%%%%%%%%%%%%%%%%%%%%%%%%%%%%%%%%%%%%%%%%%

%%% A mathematical environment in a small font as well

%\newcommand{\bea}{\begin{small}\begin{eqnarray}}
%\newcommand{\eea}{\end{eqnarray}\end{small}}

%\newcommand{\be}{\begin{small}\begin{equation}}
%\newcommand{\ee}{\end{equation}\end{small}}

\newcommand{\nn}{\nonumber \\}

%%% More useful stuff 

\newcommand{\bc}{\begin{center}}
\newcommand{\ec}{\end{center}}

\newcommand{\bt}{\begin{tabular}}
\newcommand{\et}{\end{center}}
%%%%%%%%%%%%%%%%%%%%%%%%%%%%%%%%%%%%%%%%%%%%%%%%%%%%%%%%%%%%%%%%%%%%%%%%%%%%%%%%
%%%%%%%%%%%%%%%%%%%%%%%%%%%%%%%%%%%%%%%%%%%%%%%%%%%%%%%%%%%%%%%%%%%%%%%%%%%%%%%%

\begin{document}

\noindent UFR de Physique et Ing\'enierie \hfill Universit\'e de Strasbourg

\bc
 LICENCE 3 DE PHYSIQUE --- ANN\'EE 2013/2014 \\
 
 \textbf{CALCUL SCIENTIFIQUE}\\
 
 \textbf{RAPPELS DES COMMANDES GNUPLOT} \\
\ec
 
 
\noindent\begin{tabular}{p{17.2cm}}
 \hline
 \\
\end{tabular}


\section{Utilisation du logiciel}

Pour utiliser  en invite commande, tapez simplement \verb?? dans un terminal. L'invite de commande  s'ouvre alors. Pour quitter le logiciel, utilisez les commandes \verb?quit? ou \verb?exit?. Pour obtenir de l'aide, vous pouvez taper \verb?help? suivi d'une commande.

\section{Tracer une fonction}

\noindent
Pour une fonction simple :
\begin{verbatim}
  plot sin(x)
\end{verbatim}

\noindent
Pour une fonction contenant des paramètres, on commence par définir les paramètres puis la fonction à tracer :
\begin{verbatim}
  plot a=3, sin(x*a)
\end{verbatim}

\noindent
Pour tracer une fonction en mode paramétrique, par exemple un cercle :
\begin{verbatim}
  # Pour passer en mode parametrique
  set parametric
  # Pour tracer la fonction
  plot cos(t), sin(t)
  # Pour désactiver le mode parametrique
  set parametric
\end{verbatim}

\noindent
Pour tracer une fonction en mode polaire, par exemple une spirale :
\begin{verbatim}
  # Pour passer en mode polaire
  set polar
  # Pour tracer la fonction
  plot t
  # Pour désactiver le mode polaire
  set nopolar
\end{verbatim}

\section{Tracer à partir d'un fichier de données}

\noindent
Gnuplot est capable de lire des données à partir de fichier formatés en colonnes, de la façon suivante :

\begin{center}
    \verb?example.dat?\\
\begin{tabular}{|l|}
    \hline
    \verb?# Ceci est un commentaire?\\
    \verb?# var1   var2   var3?\\
    \verb?  0.0    2.1    5.36?\\
    \verb?  0.1    2.6    5.77?\\
    \verb?  0.2    2.9    5.96?\\
    \verb?  0.3    2.7    6.84?\\
    \verb?# Ceci est un autre commentaire?\\
    \verb?  0.4    2.2    7.13?\\
    \verb?  0.5    1.9    7.32?\\
    \hline
\end{tabular}
\end{center}

\noindent
Pour tracer ces données, il suffit de taper, par exemple la deuxième colonne en abscisse et la troisième colonne en ordonnée :

\begin{verbatim}
  plot "example.dat" using 2:3
\end{verbatim}

\section{Modifier l'apparence des graphiques}

\subsection{Intervalle de représentation}

\noindent
Pour spécifier l'intervalle de représentation (par exemple $[0, 2\pi]$ en abscisse et $[-1,1]$ en ordonnée) :
\begin{verbatim}
  set xrange [0:2*pi]
  set yrange [-1:1]
  plot sin(x)
\end{verbatim}

\subsection{Type de tracés}

\noindent
Le type de tracé à réaliser peut être modifié au sein de la commande \verb?plot?. Par exemple, pour tracer des lignes au lieu de points :

\begin{verbatim}
  plot sin(x) with lines
\end{verbatim}

\noindent
Les types de tracés les plus courants sont :\\

\begin{tabular}{l|l}
    \verb?points?      & Pour utiliser uniquement des croix\\
    \verb?lines?       & Pour tracer des lignes qui relient les points\\
    \verb?linespoints? & Pour tracer à la fois les lignes et les croix\\
    \verb?dots?        & Pour tracer uniquement des petits points\\
    \verb?steps?       & Pour tracer des histogrammes\\
    \verb?boxes?       & Pour tracer des boîtes\\
\end{tabular}

\subsection{Couleur des tracés}

\noindent
De la même façon, la couleur des tracés peut être modifié dans la commande \verb?plot? via des codes couleurs. Par exemple, pour tracer en bleu, on fera :

\begin{verbatim}
  plot sin(x) linecolor 3
\end{verbatim}

\noindent
Les codes couleur sont les suivants :\\

\begin{tabular}{l|l}
    1 & Rouge\\
    2 & Vert \\
    3 & Bleu\\
    4 & Violet\\
    5 & Cyan\\
    6 & Marron\\
    7 & Orange clair \\
    8 & Orange foncé \\
\end{tabular}

\subsection{Légende}

\noindent
Il est possible de définir le titre du graphique et des axes via les commandes suivantes :

\begin{verbatim}
  set title "The title"
  set xlabel "The X axis"
  set ylabel "The Y axis""
  plot sin(x)
\end{verbatim}

\subsection{Superposition de tracés}

\noindent
Pour superposer des tracés, on peut soit lister tous les tracés dans la commande \verb?plot? :

\begin{verbatim}
  plot sin(x), "example.dat" using 2:3
\end{verbatim}

\noindent
Soit utiliser la commande \verb?replot? :

\begin{verbatim}
  plot sin(x)
  replot "example.dat" using 2:3
\end{verbatim}

\section{Utilisation de fichiers de commande}

Pour éviter d'avoir à retaper de nombreuses fois les mêmes commandes, il peut être judicieux d'utiliser des fichiers contenant déjà l'ensemble des commandes que l'on veut exécuter. On peut ensuite charger ce fichier directement au lancement de gnuplot en tapant \verb?gnuplot commands.gnu? ou bien depuis l'invite de commande en tapant \verb?load commands.gnu?. Il peut devenir utile d'ajouter la commande \verb?pause -1? dans le fichier de commande pour avoir le temps de regarder les graphiques.\\

Enfin, il est même possible de lancer l'exécution de gnuplot depuis un programme C via l'instruction \verb?system()? si vous incluez le fichier \verb?stdlib.h? :

\begin{verbatim}
  #include <stdlib.h>
  ...

  int main()
  {
      ...
      system("gnuplot commands.gnu");
      ...
  }
\end{verbatim}

\section{Exporter des tracés en fichiers latex ou png}

\noindent
Il est possible d'exporter vos graphiques directement en fichiers \verb?.tex? ou \verb?.png? de la façon suivante :

\begin{verbatim}
  plot sin(x)

  # Pour un export en fichier tex
  set terminal latex
  set output "myPlot.tex"
  replot
  set terminal x11

  # Pour un export en fichier png
  set terminal png
  set output "myPlot.png"
  replot
  set terminal x11
\end{verbatim}




\end{document}





